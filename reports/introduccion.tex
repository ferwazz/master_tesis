\chapter*{\textbf{Capítulo I: Introducción}}
\addcontentsline{toc}{chapter}{\textbf{I. Introducción}}
\fancyhead[C]{Introducción}
\setcounter{chapter}{1}


\section*{I.1 Contexto histórico}
\addcontentsline{toc}{section}{I.1 Contexto histórico}

La astronomía es considerada como la madre de las ciencias \textcolor{red}{Seg'un qui'en ??  Referencia??}, se sabe que las civilizaciones antiguas observaban los cuerpos celestes en busca de un significado para su existencia. Es fácil imaginar el por qué ocurrió así, en los inicios del hombre, sin la existencia de la contaminación lumínica, el cielo nocturno les brindaba un gran espectáculo de luces y objetos cada noche.\\

Al inicio se pensaba que el planeta Tierra era el centro del universo, más adelante con el surgimiento de la ciencia como la conocemos hoy en día, se abandonó esta idea basándose en observaciones de los cuerpos celestes en el cielo nocturno. En la era moderna, en 1930 el astrónomo Clyde Tombaugh (1906-1997) descubrió el entonces considerado noveno y más pequeño planeta del sistema solar, Plutón. Para este entonces ya se conocía la complejidad de nuestro sistema solar, compuesto por planetas, satélites naturales, asteroides, cometas y más cuerpos menores. Sin embargo siempre existió la pregunta: \textit{¿Existen otros sistemas solares como el nuestro en el universo?}.\\

En los años ochenta, un equipo encabezado por el astrónomo Gordon Walker, planearon utilizar el telescopio en la cima del volcán Mauna Kea en Hawái, para intentar encontrar la presencia de un planeta en una estrella lejana. En ese entonces ya se conocía que Júpiter completa una órbita alrededor del Sol en aproximadamente 12 años terrestres, y que debido a su gran masa, la interacción gravitacional entre entre estos dos cuerpos provoca que el Sol se mueva alrededor del baricentro del sistema, con un periodo igual al de Júpiter. La hipótesis del Dr.Walker para la búsqueda de un planeta se basaba en este fenómeno, si era capaz de medir una estrella moviéndose con un periodo orbital similar al de Júpiter, entonces podría asumir que esta estrella poseía un planeta de masa similar a la de Júpiter. Se comenzó con una muestra de 29 estrellas de tipo solar, relativamente cercanas para las cuales se intentó medir su movimiento utilizando la técnica de velocidad radial \textcolor{red}{(referencia a la SECCION DE VELOCIDAD RADIAL)}. Unos años después, se analizaron las 29 estrellas de la muestra inicial sin ningún resultado positivo.\\ 

No fue hasta 1995 cuando Michel Mayor y Didier Queloz encontraron una estrella con un movimiento extraño. 51 Pegasi, una estrella de una muestra de 142 que habían estado bajo observación, con una temperatura similar a la del Sol (5768 K), a unos 50 ly \textcolor{red}{mejor decir en espa~nol an~os luz} (1 ly = 9.461e+15 m) de distancia. Los datos de la observación revelaron que 51 Pegasi tenía un periodo órbital de 4.2 días, lo cual resultó asombroso y difícil de creer para el Dr. Mayor. El hecho de que un planeta como Júpiter completara una órbita alrededor de una estrella tipo solar en tan solo 4.2 días parecía increíble. Después de corroborar las observaciones, finalmente el 23 de Noviembre de 1995 se publicó el hallazgo de un planeta de una masa similar a la de Júpiter \cite{mayor1995jupiter}. En 2019 Michel Mayor y Didier Queloz fueron galardonados con el premio Nobel de Física por el descubrimiento del primer planeta alrededor de una estrella tipo solar, lo que actualmente conocemos como exoplaneta.\\

En los meses siguientes al descubrimiento de 51 Pegasi b en 1995, se encontraron planetas con periodos similares, como \cite{Marcy_1996} y \cite{butler1996planet}, descubiertos por el equipo conformado por Geoffrey Marcy y Paul Butler, los cuales competían con el Dr. Mayor en la búsqueda del primer exolplaneta, \textcolor{red}{NO entiendo esta frase, quienes fueron los primeros??también Marcy y Butler fueron los primeros en confirmar el descubrimiento de 51 Pegasi b}. En los años siguientes se siguieron descubriendo planetas en estrellas de tipo solar utilizando esta misma técnica, de hecho, en 2003 el Dr. Walker y un equipo internacional, confirmó la existencia de un planeta del tamaño de Júpiter en la muestra inicial de 29 estrellas que fue estudió en los 80's \cite{hatzes2003planetary}, \cite{walker2012first}.\\

Años después, David Charbonneau y Gregory W. Henry estudiaban una de las estrellas de la muestra original del Dr.Mayor, para la cual ya se había confirmado la existencia de un exoplaneta utilizando mediciones de velocidad radial. Sin embargo, estaban en busca de variaciones de brillo debido al paso del planeta frente a la estrella, este fenómeno es conocido como tránsito. En 1999 publicaron el hallazgo del tránsito de HD209458b \cite{charbonneau1999detection}, un planeta 60\% más grande que Júpiter, con una órbita de 3.5 días, el cual provocó una caída del 1.7\% en el brillo de HD209458. Este mismo año se publicó el descubrimiento del primer sistema múltiple, alrededor de \textit{Upsilon Andromedae} \cite{marcy1999three}, 3 planetas de masas similares a la de Júpiter que orbitan una estrella de 1.28 $M_{\odot}$.\\

Estos primeros planetas en ser descubiertos tenían algo en común, eran de una masa similar a la de Júpiter con periodos orbitales cortos, los cuales se denominan Júpiters calientes (\textit{Hot Jupiters}) o tenían órbitas muy excéntricas como el caso de HD222582b \cite{vogt2000six}, ambas características son bastante diferentes a lo que conocemos en nuestro sistema solar, lo cual originaba más preguntas: \textit{¿por qué los sistemas planetarios descubiertos son tan diferentes al nuestro?, y entonces, ¿cómo se formó nuestro sistema solar?}, preguntas para las cuáles, aún no tenemos una respuesta clara.\\ 

\section*{I.2 Proyectos enfocados en la búsqueda de exoplanetas por el método de tránsito}
\addcontentsline{toc}{section}{I.2 Proyectos enfocados en la búsqueda de exoplanetas por el método de tránsito} 	

Los primeros exoplanetas fueron descubiertos utilizando el método de velocidad radial. Para llevar a acabo una medición de este tipo, se necesita de un espectrógrafo de alta resolución y mucho tiempo de monitoreo a una sola estrella, para obtener las mediciones de velocidad que corroboren la existencia de anomalías gravitacionales provocadas por un planeta. Por otra parte, con el descubrimiento del tránsito de HD209458b, el método del tránsito se volvió el más exitoso para buscar planetas que orbitan otras estrellas. Una de las ventajas más importantes es que se puede monitorear la intensidad de muchas estrellas al mismo tiempo, como es el caso del proyecto TAOS 2, el cual observará alrededor de 10,000 estrellas simultáneamente.\\

Desde entonces y hasta la actualidad, se han desarrollado cerca de 50 proyectos enfocados en la búsqueda de exoplanetas, midiendo variaciones periódicas en la luminosidad provocadas por un tránsito exoplanetario. Enseguida se mencionan solo de algúnos de los proyectos más exitosos.\\



\subsection*{I.2.1 CoRoT}
\addcontentsline{toc}{subsection}{I.2.1  CoRoT} 

Para mediados de 2006, todos los planetas descubiertos orbitando estrellas lejanas, eran del tipo \textit{Hot Jupiters}, gigantes gaseosos de más de veinte veces el radio del planeta Tierra (>$20R_{\Earth}$). El 27 de diciembre de 2006, se lanzó el primer satélite enfocado en la búsqueda de exoplanetas por el método de tránsito, el \textit{COnvection ROtation and planetary Transits} (CoRoT), dedidcado a la búsqueda de planetas rocosos un poco más parecidos a la Tierra \cite{aigrain2007corot}, \cite{barge2007corot}, cosa que desde la superficie terrestre no era posible debido a los efectos de distorsión atmosférica.\\

Originalmente CoRoT fue pensado como un proyecto de astrosismología\footnote{La asteroseismología estudia la estructura interna de las estrellas mediante la interpretación de sus espectros como ondas sísmicas.}, sin embargo, se acordó el incluir la búsqueda de tránsitos de exoplanetas como su nombre lo indica. En mayo del 2007 se reportó el descubrimiento del primer tránsito de un exoplaneta no conocido, el denominado CoRoT-1b \cite{barge2008transiting}. \\

La misión CoRoT estaba equipada con un espejo primario de 27 cm  y una cámara de campo amplio que funcionaba en el espectro visible. El telescopio contenía un deflector de rendimiento extremadamente alto para minimizar la luz directa en el detector. La luz capturada por el telescopio se dirige al plano focal, que consta de cuatro CCD con 2048x4096 píxeles dispuestos en un patrón cuadrado con una escala de placa de 2.32"/pix ("/pix = segundos de arco por pixel). Dos de los CCD estaban dedicados a la astrosismología, mientras que los dos restantes al programa exoplanetas. Las curvas de luz de las estrellas son producidas por el software a bordo que utiliza fotometría de apertura y posteriormente se trasmiten a la Tierra cada 512 segundos. Este instrumento es clasificado como satélite de clase mini (630 kg), que se colocó en una órbita polar con una altitud de 896 km y un período de 6,714 s.\\

Al final, se recolectaron 163,000 curvas de luz, con más de 500 candidatos a tránsitos, de los cuales solo 34 (7\%) fueron confirmados como planetas \cite{moutou2013corot}. Tuvo contribuciones relevantes como el descubrimiento del primer exoplaneta de tipo rocoso el CoRoT-7b \cite{leger2009transiting}, el estudio de estructuras internas de estrellas y nos proveyó de restricciones para las teorías de formación y evolución de sistemas planetarios.

\subsection*{I.2.2 HATNet y HATSouth}
\addcontentsline{toc}{subsection}{I.2.2 HATNet y HATSouth} 

Los proyectos \textit{Hungarian-made Automated Telescope Network} (HATNet) y \textit{Hungarian-made Automated Telescope Network-South} (HATSouth) conforman una red de 13 telescopios distribuidos en 3 continentes, optimizados para la búsqueda de tránsitos de exoplanetas \cite{bakos2002system}, \cite{bakos2004wide}, \cite{bakos2013hatsouth}. \\

Esta red de telescopios intenta detectar y caracterizar planetas extrasolares, utilizando telescopios pequeños totalmente automatizados, constituidos con cuatro telescopios f/2.8 de 0.18 m de diámetro en una montura común que cubren un área de ocho grados cuadrados en el cielo, usando cuatro cámaras con CCD's 4K con iluminación frontal y filtros \textit{Sloan}, para obtener una escala de 3.7 minutos de arco por pixel (arcmin/pixel). Como se menciono anteriormente, las principales ventajas de los estudios de tránsito son los grandes campos que se pueden observar, lo que se traduce en un gran número de estrellas observadas simultáneamente, sin embargo, debido a las características ópticas de los telescopios, sus objetivos se limitan a estrellas brillantes (magnitud $V \leq 13$).\\

Para HATNet y HATSouth, en total, se han realizado observaciones de seguimiento para 3,200 de los 4,200 candidatos. Se tienen 140 objetos subestelares confirmados de los cuales 128 son exoplanetas, el más reciente publicado en diciembre del 2018 \cite{bakos2018hats}. Se concluyó que 2,300 son falsos positivos o falsas alarmas. Los candidatos restantes requieren más observaciones de seguimiento.\\

\subsection*{I.2.3 SuperWASP}
\addcontentsline{toc}{subsection}{I.2.3 SuperWASP}

El proyecto \textit{Wide Angle Search for Planets} (SuperWASP) es una red de observadores internacionales dedicados a la búsqueda de exoplanetas, uno de lo proyectos con telescopios terrestres líderes en el área con 185 exoplanetas confirmados \cite{pollacco2006wasp}. El proyecto utiliza dos observatorios robóticos ubicados en el Observatorio del Roque de los Muchachos en la isla de La Palma en las Islas Canarias (SuperWASP-North) y en el Observatorio Astronómico Sudafricano (SuperWASP-South) \textcolor{red}{mencionar el tamanio de los telescopios}, lo que permite observaciones en ambos hemisferios. SuperWASP-North y SuperWASP-South comenzaron con sus actividades en 2003 y 2005 respectivamente, poco tiempo después se reportó el descubrimiento de dos planetas de 1.33 y 1.26 $R_{Jup}$ \cite{cameron2007wasp}.\\


Tanto \textit{SuperWASP-North} como \textit{SuperWASP-South} poseen monturas ecuatoriales con 8 cámaras de campo amplio. Cada cámara está equipada con un CCD que consta de 2048x2048 pixeles, con un tamaño de 13.5 $	\mu m$ . El campo de visión de cada cámara cubre un área de 64 grados cuadrados con una escala angular de 13.7"/pix. Las cámaras observan campos extremadamente amplios, y el conjunto completo de 8 cámaras es capaz de monitorear hasta 480 grados cuadrados en el cielo, siendo considerablemente más grande que otros telescopios convencionales. Durante los primeros 6 meses de observaciones, el SuperWASP-North proporcionó más de 6.7 millones de curvas de luz para estrellas con $V \sim 7.5-15$ mag.\\


Los datos recopilados por las cámaras son automáticamente procesado por un pipeline, que incluye varios pasos. En el primer paso, el pipeline corrige las imágenes eliminando errores causados por el detector o defectos ópticos. Posteriormente, las imágenes se reducen al construir y aplicar las imágenes de calibración \textit{Bias master}, \textit{Darks} y \textit{Flats}, para cada noche de observación. La fotometría de apertura se utiliza para medir el flujo estelar y construir las curvas de luz de cada fuente individual que se logra identificar en los pasos anteriores. Finalmente, la detección de tránsitos periódicos se realiza aplicando el algoritmo Box Fitting Algorithm a las curvas de luz \cite{kovacs2002box}. Para más detalles sobre la selección de candidatos, ver \cite{collier2006fast}.\\

\subsection*{I.2.4 Telescopio espacial \textit{Kepler}}
\addcontentsline{toc}{subsection}{I.2.4 Telescopio espacial \textit{Kepler}}

La misión Kepler es por mucho, el proyecto más exitoso dedicado a la búsqueda de exoplanetas por el método del tránsito \cite{borucki2010kepler}. La misión fue diseñada con el objetivo de encontrar planetas rocosos, con masas y tamaños similares a la Tierra en la zona de habitabilidad. La misión se lanzó el 6 de marzo del 2009, tres años después se
anunció el descubrimiento del primer planeta rocoso en la zona habitable \textit{Kepler-22b} \cite{borucki2012kepler}, un planeta con $2.4R_{\Earth}$ el cual se denominó como SuperTierra. Conforme avanzaba el proyecto, más descubrimientos importantes se anunciaban, con esto, nos dimos cuenta que los sistemas múltiples y planetas en la zona habitable son más comunes de lo que pensábamos.\\

La misión estaba diseñada para observar constantemente alrededor de 150,000 estrellas de secuencia principal con magnitudes menores \textcolor{red}{no son mayores??} a $V=14$ mag en la misma región en el cielo, en la dirección de las constelaciones de \textit{Cygnus}, \textit{Lyra} y \textit{Draco}. La sonda espacial \textit{Kepler} está equipada con un telescopio Schmidt 24 de 0.95 m de apertura y un espejo primario de 1.4 metros de diámetro. El detector consiste en una matriz de 42 CCD’s que se leen cada 6 segundos para evitar la saturación. Cada CCD tiene un tamaño de 50x25 mm con 2200x1024 pixeles, cubriendo un área en el cielo de 105 grados cuadrados.\\

Los datos obtenidos se almacenaban en la sonda espacial y enviados a la Tierra cada mes. Los datos se recopilan en el Kepler Mission Science Operations Center y eran procesados mediante el pipeline de \textit{Science Processing Pipeline} \cite{jenkins2010overview}, el cual incluye consiste en distintos procesos: calibración de nivel, análisis fotométrico, extracción y corrección de curvas de luz, algoritmo de detección de tránsitos, selección de candidatos y validación de candidatos. Las curvas de luz se extraen mediante fotometría de apertura y posteriormente se corrigen eliminando las firmas correlacionadas con variables instrumentales, como los desplazamientos apuntado y los cambios de foco.
Además, se eliminan los valores atípicos (outliers) y las discontinuidades causadas por los cambios de sensibilidad de píxeles. La detección de tránsitos periódicos se lleva a cabo mediante el \textit{Transiting Planet Search (TPS)}, un algoritmo que realiza una estimación el espectro de potencia del ruido de observación en función del tiempo \cite{jenkins2010transiting}.\\

A lo largo de su operación se realizaron muchos descubrimientos importantes, como un sistema planetario orbitando una estrella gigante roja \cite{charpinet2011compact}, \textit{Kepler-47} un sistema de dos planetas, uno de ellos en la zona habitable que orbitan un sistema estelar binario \cite{orosz2012kepler}, \textit{Kepler-90} un sistema de 8 planetas pequeños, todo el sistema es más pequeño que la órbita del planeta Tierra \cite{shallue2018identifying}, el octavo planeta se confirmó utilizando un algoritmo de Deep Learning desarrollado por miembros de Google, y muchos más. El 30 de Octubre del 2018 la sonda quedó sin combustible y oficialmente el final del proyecto, esta se trasladó a una orbita segura. Después de casi 10 años, se observaron cerca de 530,506 estrellas, 2662 planetas confimados, 61 supernovas documentadas, 2878 estrellas binarias eclipsantes confirmadas y alrededor de 3000 publicaciones relacionadas con el proyecto, hacen del telescopio espacial Kepler el proyecto más exitoso en la historia de la búsqueda de nuevos planetas, gracias al cual, ahora sabemos que seguramente que existen más planetas que estrellas en el universo.\\

\subsection*{I.2.5 Satélite \textit{TESS}}
\addcontentsline{toc}{subsection}{I.2.5 Satélite \textit{TESS}}

El satélite \textit{TESS} del inglés \textit{Transiting Exoplanet Survey Satellite} es el sucesor del telescopio \textit{Kepler}. El objetivo principal de \textit{TESS} es detectar planetas pequeños en estrellas brillantes relativamente cercanas, utilizando el método del tránsito. La primera gran diferencia con \textit{Kepler}, es que \textit{TESS} realizará un monitero de aproximadamente el 90\% de la bóveda celeste, con un alcance cerda de 200 ly de distancia. La gran ventaja de esto, es que al estudiar principalmente estrellas brillantes, estas pueden ser observadas fácilmente por telescopios en tierra y otros telescopios espaciales. 

Cuenta con 4 cámaras identicas de campo aplio (24$^{\circ}$ x 24$^{\circ}$) cada una, lo que resulta en un campo de 24$^{\circ}$ x 96$^{\circ}$ (un total de 3200 grados cuadrados) y un radio focal f/1.4. A su vez, cada cámara está equipada con lentes para prevenir la dispersión de la luz provenientes de la Tierra y la Luna y un arreglo de 4 CCD’s de 4096x4094 pixeles con una escala angular de 21 arcsec/pixel.

Los CCD's de las cámaras producen constantemente imágenes de 2 segundos de exposición, las cuales se combinan de 2 diferentes maneras. Primeramente se suman 900 imágenes para simular una exposición continua de 30 minutos del campo completo llamado \textit{ Full-Frame Images (FFIs)}, también se suman grupos de 60 imágenes como exposiciones de 2 minutos catalogadas como "\textit{Postage Stamps}", las cuales son preprocesadas y recortadas en secciones de 10x10 pixeles centradas en estrellas de interés. \cite{ricker2014transiting}.

El plan es que el proyecto tenga una duración de 2 años, en el cual se observe completamente el hemisferio sur el primer año y el hemisferio norte en el segundo. Se utiliza un método de apuntado denominado "\textit{stop and stare}", lo que significa que observará un mismo campo completo (24$^{\circ}$ x 96$^{\circ}$) por una duración de 24 dias, para despúes rotar 27.7$^{\circ}$ hacia el este y repetir el proceso de observación. Con esta estrategia se necesitarán un total de 26 sectores para cubrir el 90\% del cielo. \cite{schliegel2017tess}

\textit{TESS} está enfocado en estrellas de tipo espectral F5 a M5, es decir estrellas relativamente pequeñas en las cuales es más sencillo encontrar planetas pequeños del tipo rocoso. \textit{CoRoT} y \textit{Kepler} tenían mayor presición fotométrica capaz de detectar planetas más pequeños, sin embargo, la desventaja es que estos planetas no son viables para su estudio continuo debido al poco brillo de sus estrellas anfitrionas, en las cuales sería practicamente imposible realizar estudios sobre sus propiedades atmosféricas. \cite{sullivan2015transiting} 

El satélite fue lanzado el 18 de abril de 2018 y entro en complata operación el 25 de Julio del mismo año. Al momento de la escritura de este trabajo, \textit{TESS} a confirmado 37 nuevos planetas y se tienen alrededor de 1600 candidatos, entre ellos el primer planeta del tipo rocoso, en la zona de habitabilidad, con un radio de $1.22R_{\Earth}$ y una órbita de 37.42 dias. \cite{gilbert2020first}

Se espera que en los 2 años de operación, la sonda \textit{TESS} descubra alrededor de 20,000 planetas, con un estimado de 500 con un radio menor a $2R_{\Earth}$.


\textcolor{red}{falta dedicarle un par de p'arrafos a TAOS~II, adem'as vale la pena mencionar si hay o no traslape con el 'area de cielo de los otros proyectos, asi como magnitudes y enfatizar ue TAOS~II no es para b'usqueda de exoplanetas... }

\section*{I.3 Objetivos, Hpi'otesis y estructura de la tesis}
\addcontentsline{toc}{section}{I.3 Objetivos y estructura de la tesis}


\textcolor{red}{Por favor escribe los objetivos y la hip'otesis antes de continuar con lo dem'as}










