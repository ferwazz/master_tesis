\chapter*{\textbf{Capítulo IV: Análisis de Resultados}}
\addcontentsline{toc}{chapter}{\textbf{IV. Análisis de Resultados}}
\chead{Análisis de Resultados}

\section*{IIII.1 Identificación y eliminación de ruido en curvas de luz}
\addcontentsline{toc}{section}{IIII.1 Identificación y eliminación de ruido en curvas de luz}

Primeramente presentamos los resultados de las metodologías para la mejora de la SNR en las curvas de luz obtenidas en las temporadas de observación, y las curvas con tránsitos simulados.

\subsection*{IIII.1.2 Datos observacionales}
\addcontentsline{toc}{subsection}{IIII.1.2 Datos observacionales}




\subsection*{IIII.1.3 Curvas de luz simuladas}
\addcontentsline{toc}{subsection}{IIII.1.3 Curvas de luz simuladas}

Se creo una base de datos de ruidos simulados con semilla para asegurar reproducibilidad de los resultados. 




\section*{IIII.2 Determinación de candidatos a tránsitos}
\addcontentsline{toc}{section}{IIII.2 Determinación de candidatos a tránsitos}

\subsection*{IIII.2.1 Datos observacionales}
\addcontentsline{toc}{subsection}{IIII.2.1 Datos observacionales}

\subsection*{IIII.2.2 Simulaciones}
\addcontentsline{toc}{subsection}{IIII.2.2 Simulaciones}


